% !TEX root = ../../../main/aws_perfectoid.tex
\newpage
\section{Bhargav Bhatt: $p$-adic Hodge Theory}
\subsection{Lecture 1}

Hodge Decomposition

$X/\C$ smooth projective curve

\begin{thm}
There exists a natural isomorphism $H^n(X\an, \Q) \otimes \C \cong \bigoplus_{i+j=n} H^i(X, \Omega^j_{X/\C})$
\end{thm}

\begin{ex}
$X=E$ elliptic curve over $\C$ then $E= \C/\Lambda$
\end{ex}


\begin{thm}
	\[
	\begin{tikzcd}
	H^0(X,\Omega^1_X) \arrow[hook]{r} & H^1(X\an,\Q) \otimes \C \\
	\C \omega \arrow[draw=none]{u}[sloped,auto=false]{=} & \Hom(\Lambda, \C) \arrow[draw=none]{u}[sloped,auto=false]{=} \\
	\omega \arrow[mapsto]{r} & \ds r \in \Lambda \mapsto \int_\gamma \omega
	\end{tikzcd}
	\]
\end{thm}


Highly Transcendental 


\begin{cor}
Say $f: X \to Y$ of smooth projective variety and $f^*: H^n(Y, \Q) \ma{\sim} H^n(X,\Q)$ then $H^i(X,\Omega^j_X) \rma{\sim} H^i(Y,\Omega_Y^j): f^*$ for all $i+j=n$
\end{cor}


\cetale Cohomology

Say $X$ is a scheme
$A \in \{ \Z/n\Z, \Z_p, \Q_p \}$
Grothendieck then $H^*(X_\et, A)$ algebraically defined 

\begin{thm}[Artin]
Let $X/\C$ be a variety then $H^*(X_\et,A) \ma{\sim} H^*(X\an,A)$.
\end{thm}

Upshot: Say $X$ is defined over $\Q$

Theorem $+$ epsilon there exists a natural action $G_\Q$ on $H^*(X\an,A)$.


\begin{ex}
\begin{enumerate}[(i)]
\item $X=E$ elliptic curve over $\C$ but defined over $\C$. Therefore, $E= \C/\Lambda$.
	\[
	\begin{aligned}
	H^1(X\an, \Z/n\Z)&= \Hom(H_1(X\an, \Z/n\Z), \Z/n\Z) \\
	&\cong \Hom(\Lambda, \Z/n\Z) \\
	&\cong E[n]^\vee
	\end{aligned}
	\]
Theorem then $E[n]$ is defined over $\ov{\Q}$. Get action $G_\Q$ on $E[n]$.

Set $T_pE= \plim E[p^n]$. Therefore, get a constant $G_\Q$-action on $T_pE$ if and only if dual to the $G_\Q$-action on $H^1(X\an, \Z_p)$. 

\item $X=\G_m$ some analysis shows
	\[
	H^1(\G_n\an, \Z/n\Z) \cong \mu_n^\vee
	\]
Set $\Z_p(1)= \plim_n \mu_{p^n}$.
Therefore, get $G_\Q$-action on $\Z_p(1)$ if and only if $G_\Q$-action on $H^1(X\an, \Z_p)$


Notation: For any $\Z_p$-algebra $R$, set $R(i):= R \otimes_{\Z_p} \Z_p(1)^i$

Note: if $G_\Q$ acts on $R$, it also acts on $R(i)$


\item $X= \P^1$
	\[
	H^2(\P\an, \Q_p) \cong H^1(\G_m\an, \Q_p) \cong \Q_p(-1)
	\]
as $G_\Q$-modules
More generally, if $X$ smooth projective of dimension $d$, then 
	\[
	H^{2d}(X\an, \Q_p) \cong \Q_p(-d)
	\]
\end{enumerate}
\end{ex}


Hodge-Tate Decomposition

Fix a prime $p$, $K/\Q_p$ finite extension
	\[
	K \subset \ov{K} \subset \hat{\ov{K}}= \C_p
	\]
$G_K= \Gal(\ov{K}/K)$ acts on $\ov{K}$ and $G_K$ acts on $\C_p$.


\begin{thm}[Hodge-Tate Decomposition]
Say $X/K$ is a smooth projective variety, then there is a natural $G_K$-equivariant isomorphism
	\[
	H^n(X_{\ov{K}}, \Q_p) \otimes_{\Q_p} \C_p \cong \bigoplus_{i+j=n} H^i(X, \Omega^j_{X/K} \otimes_K \C_p(-1)
	\]
where $G_K$ acts in the natural way on both sides. 
\end{thm}

To use this theorem use

\begin{thm}[Tate]
Fix $i \neq j \in \Z$
	\[
	\begin{aligned}
	\Hom_{G_K}(\C_p(i), \C_p(j))&= 0 \\
	\Ext_{G_K}^1(\C_p(i),\C_p(j))&= 0
	\end{aligned}
	\]
\end{thm}


\begin{ex}
\begin{enumerate}[(i)]
\item $X= \P^1/K$, $n=2$
	\[
	\begin{tikzcd}
	H^2(X_{\ov{K}}, \Q_p) \otimes \C_p \arrow[draw=none]{r}[sloped,auto=false]{\cong} & \left( H^2(X,\O_X) \otimes \C_p \right) \oplus \left( H^1(X,\Omega_X^1) \otimes \C_p(-1) \right) \oplus \left(H^0(X,\Omega_X^1) \otimes \C_p(-1) \right) \\
	\O_p(-1) \otimes \C_p \arrow[draw=none]{u}[sloped,auto=false]{=} & 0 \oplus \C_p(-1) \oplus 0 \arrow[draw=none]{u}[sloped,auto=false]{=} \\
	\C_p(-1)
	\end{tikzcd}
	\]

\item $X=E$ elliptic curve over $K$
	\[
	\begin{tikzcd}
	H^1(X_{\ov{K}},\Q_p) \otimes \C_p \arrow[draw=none]{r}[sloped,auto=false]{=} & \left(H^1(X,\O_X) \otimes_K \C_p \right) \oplus \left(H^0(X,\Omega_X^1) \otimes_K \C_p(-1) \right) \\
	T_p(E)^\vee \otimes \C_p \arrow[draw=none]{u}[sloped,auto=false]{\cong} & \C_p  \oplus \C_p(-1) \arrow[draw=none]{u}[sloped,auto=false]{=}
	\end{tikzcd}
	\]
$\Lie(E^r) \otimes \C_p$
$\Lie(E)^\vee \otimes \C_p(-1)$
\end{enumerate}
\end{ex}

\begin{cor}
$X/K$ smooth projective then 
	\[
	H^i(X,\Omega^j_{X/K}) \cong \left( H^{i+j}(X_{\ov{K}}, \Q_p) \otimes \C_p(j) \right)^{G_K}
	\]
\end{cor}

\begin{rem}
Ito used this cor to reprove
\end{rem}

\begin{thm}
$X,Y$ Calabi-Yao varieties over $\C$, $X \stackrel{\text{bir}}{\sim} Y$, then $\dim H^j(X,\Omega^i_{X/K})=:h^{i,j}(X) = h^{i,j}(Y)$
\end{thm}

\begin{rem}
There exists a good variant for general $X$
\end{rem}


Hodge-Tate Spectral Sequence

Use perfectoid spaces to prove

\begin{thm}[HT,SS]
$C/\Q_p$ complete and algebraically closed
$X/C$ proper smooth rigid-analytic space
then there exists an $E_2$ spectral sequence
	\[
	E_2^{ij}: H^i(X,\Omega^j_{X/C})(-j) \ma{} H^{i+j}(X,\Q_p) \otimes \C
	\]
then get Hodge-Tate filtration on $H^n(X,\Q_p) \otimes \C$
\end{thm}


\begin{rem}
\begin{enumerate}[(i)]
\item HT SS is functorial
then if $X$ is defined over $K$ (with $K/\Q_p$ finite) then Tate's Theorem then get HT decomposition for $X$.

\item the HT SS always degenerates (Conrad-Gabbor) but not canonically so:

\begin{ex}
Say $X= E$ elliptic curve.
HT SS then low degree SES
	\[
	0 \ma{} H^1(X, \O_X) \ma{} H^1(X,\Q_p) \otimes \C_p \ma{} H^0(X,\Omega_X^1)(-1) \ma{} 0
	\]
maps go the wrong way
cannot choose a splitting that varies well in family
\end{ex}
\end{enumerate}
\end{rem}











